\documentclass{article}
\usepackage[utf8]{inputenc}
\usepackage{natbib}
\usepackage[french]{babel}
\usepackage{pdfpages}

\title{Rapport du projet de Programmation Système}
\date{\today}

\usepackage{graphicx}
\usepackage[T1]{fontenc}
\usepackage{hyperref} % 		Ajouter des liens
\usepackage{color} % 			Gestion des couleurs
\usepackage[normalem]{ulem} % 	Underline

\usepackage{comment}%			Commenter

\usepackage{listingsutf8}% 		Code informatique
\definecolor{mGreen}{rgb}{0,0.6,0}
\definecolor{mGray}{rgb}{0.5,0.5,0.5}
\definecolor{mPurple}{rgb}{0.58,0,0.82}
\definecolor{backgroundColour}{rgb}{0.95,0.95,0.92}
\lstset{
	tabsize=4,
    	extendedchars=true,
    	breaklines=true,
    	keepspaces=true,
	inputencoding=utf8,
    	extendedchars=true,
    	literate=%
                {é}{{\'e}}{1}%
                {è}{{\`e}}{1}%
                {à}{{\`a}}{1}%
                {ç}{{\c{c}}}{1}%
                {œ}{{\oe}}{1}%
                {ù}{{\`u}}{1}%
                {É}{{\'E}}{1}%
                {È}{{\`E}}{1}%
                {À}{{\`A}}{1}%
                {Ç}{{\c{C}}}{1}%
                {Œ}{{\OE}}{1}%
                {Ê}{{\^E}}{1}%
                {ê}{{\^e}}{1}%
                {î}{{\^i}}{1}%
                {ô}{{\^o}}{1}%
                {û}{{\^u}}{1}%
                {ë}{{\¨{e}}}1
                {û}{{\^{u}}}1
                {â}{{\^{a}}}1
                {Â}{{\^{A}}}1
                {Î}{{\^{I}}}1
}
\lstdefinestyle{CMakeStyle}{
    language={},
  	backgroundcolor=\color{backgroundColour},
    numberstyle=\tiny\color{mGray},
    basicstyle=\footnotesize,
    numbers=left,
    numbersep=5pt,
  	frame=single,
}
\lstdefinestyle{CStyle}{
    backgroundcolor=\color{backgroundColour},
    commentstyle=\color{mGreen},
    keywordstyle=\color{magenta},
    numberstyle=\tiny\color{mGray},
    stringstyle=\color{mPurple},
    basicstyle=\footnotesize,
    breakatwhitespace=false,
    breaklines=true,
    captionpos=b,
    numbers=left,
    numbersep=5pt,
    language=C
}



\begin{document}

\begin{titlepage}
 \begin{sffamily}
  \begin{center}
            \includegraphics[scale=0.04]{img/ubx-logo.png}
            \\[2cm]
        
    
    {\huge \bfseries Rapport de Projet technologique\\[0.5cm] }

    \rule{\linewidth}{.5pt}
    \\[2cm]

    \begin{minipage}{0.4\textwidth}
      \begin{flushleft} \large
        \author{}Etudiants :\\
        	Marc \textsc{Cerutti}\\
      \end{flushleft}
    \end{minipage}

    \vfill

    {\large \today}

  \end{center}
  \end{sffamily}
  
\end{titlepage}

\newpage

\tableofcontents


\begin{comment}
%CONSIGNES


\end{comment}


% - Introduction ------ %
\newpage
\part{Introduction}

\newpage
% - Developpement ----- %
\part{Développement}
% ----- Partie A ------ %
\section{Partie A}

% ----- Partie B ------ %
\newpage
\section{Partie B}

% - Conclusion -------- %
\newpage
\part{Conclusion}


\newpage

\appendix

\part{Annexes}

\listoffigures

\newpage
Rapport du groupe CERUTTI
Le 05 novembre 2018

Interface graphique:
	-Rajout d'une check box pour remettre à l'image d'origine automatiquement dans la fenêtre principal.
	-Rajout de l'affichage de l'approximation de l'efficacité des fonctions.
	-Mise a jour du code des boutons, des menus, et des fenêtres pour implémenter ces nouvelles mécaniques.
	-Rajout d'une fenêtre pour les paramètres de StereoBM.
	-Rajout des boutons de Sobel et Flou gaussien.
	
Analyse d'images:
	-Amélioration des algorithmes pour prendre en charge les images en GrayScale, permet d'enlever les conversions
		redondantes, et l'utilisation des fonctions de manière successives facilité.
	-Rajout de showMatrice(cv::Mat mat) dans ImageAnalyser pour le débuguage.
	-Rajout de computeEfficiency(double time, func, args) pour avoir une approximation de l'efficacité des algorithmes.
	-Rajout des fonctions Sobel et Flou gaussien.

Mise en place d'un exécutable pour les tests d'analyse d'images.

Pistes d'améliorations:
	-Implémentation des tests
	-Recherche des meilleurs paramètres pour algorithmes stereo
	-Factorisation du code pour les fenêtres BM et SGBM
	-Recherche sur la possible utilisation de threads
	-SteroVar non essayé openCv
	-Recherche analyse de flux vidéo
	-Recherche object tracking et template



////

Pas d'image dans git
Signaler fuites mémoire, meme biblotheque
Sobel n'est pas juste la fonction, voire voca, faire un gradient
Diparity map, négatif ?
Voire pour normalisation
Image inversée ?
Trop de découpage 
On voit pas grayscale
Laisser le code cvtColor ne faccilite pas le code
Mettre plus de références

PROCHAINE SEANCE
Simplifier le code
Rennomer les variables avec convention (code et fichiers) et commenter le code
Voire carte disparité probleme


AMELIORATIONS
Prévoir quoi faire avec le matériel

RENDU FINAL
Parametres
Carte de disparité
Carte de profondeur
Rapport
Avec graphiques performances
Tests
Soutenance


\newpage

\end{document}
